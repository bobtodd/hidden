\documentclass[12pt]{article}

\usepackage{amsmath,amsthm}
\usepackage{bm} % bold-math, for getting boldface Greek letters
\usepackage{bbm} % to get blackboard-bold fonts, including double-stroke 1
\usepackage{booktabs} % for nice tables
\usepackage{enumitem} % for modifying list formats

\usepackage[colorinlistoftodos]{todonotes}

\usepackage{color}
\definecolor{rltred}{rgb}{0.75,0,0} % used below in hyperref
\definecolor{rltgreen}{rgb}{0,0.5,0}
\definecolor{rltblue}{rgb}{0,0,0.75}

% \usepackage[comma,authoryear]{natbib} % In case we change bibliography style
% \usepackage{hypernat} % handles conflicts between hyperref and natbib
\usepackage[pdftex, % comment in if pdftex
        %xetex, % comment in if xetex
        colorlinks=true,
        urlcolor=rltblue, % \href{...}{...} external (URL)
        filecolor=rltgreen, % \href{...} local file
        linkcolor=rltred, % \ref{...} and \pageref{...}
        pdftitle={For Whom the Bell Tunnels},
        pdfauthor={Takahide Okabe, Todd B. Krause},
        pdfsubject={Quantum Mechanics},
        pdfkeywords={physics, quantum mechanics, hidden variables},
        %pdfadjustspacing=1,
        pagebackref,
        pdfpagemode=None,
        %bookmarksopen=true,
        plainpages=false, % plainpages and pdfpagelabels to
        pdfpagelabels, % get book class pagenumbers right.
        %hyperfootnotes=false, % ledmac does crazy things with
        %debug=true% % footnotes, and hyperref can't
        ]{hyperref} % handle it.

\newtheorem{thm}{Theorem}
\newtheorem{lem}[thm]{Lemma} % including [thm] this way keeps Theorem and Lemma on the same numbering system
\newtheorem{prop}[thm]{Proposition}

\newcommand{\heq}{\stackrel{\heartsuit}{=}}

\title{For Whom the Bell Tunnels}
\author{T `n' T}

\begin{document}
\maketitle
\listoftodos

\section{Notes on Expectation Values}

Before launching into a discussion of \cite{Bell1966}, we should review a few points about expectation values.\footnote{See the \href{http://en.wikipedia.org/wiki/Expectation_value_\%28quantum_mechanics\%29}{Wikipedia article} for a quick reference.} If we have a system in a \emph{pure} state $|\psi\rangle$ and we seek to measure a quantity with associated operator $\hat{A}$, then the expectation value of $\hat{A}$ \emph{in that state} is
\begin{displaymath}
  \langle \hat{A} \rangle_{\psi} := \langle \psi | \hat{A} | \psi \rangle.
\end{displaymath}
Of course, if have some complete orthogonal basis of states $\{ \phi_n \}$, we may expand the pure state $\psi$ in this basis as
\begin{displaymath}
  |\psi\rangle = \sum_{n} a_n |\phi_n \rangle,
\end{displaymath}
and then the expectation value becomes
\begin{displaymath}
  \langle \psi | \hat{A} | \psi \rangle
  = \langle \psi | \hat{A} \sum_{n} a_n | \phi_n \rangle
  = \sum_{n} a_n \langle \psi | \hat{A} | \phi_n \rangle.
\end{displaymath}

The situation is slightly different if we do not have a pure state, but rather a \emph{mixed} state. In such cases we employ a density operator $\hat{\rho}$ composed of the various individual pure states $| \psi_k \rangle$, expressed as follows:
\begin{displaymath}
  \hat{\rho} := \sum_k \rho_k | \psi_k \rangle \langle \psi_k |.
\end{displaymath}
Then the expectation value of $\hat{A}$ in this collection of mixed states becomes
\begin{displaymath}
  \langle \hat{A} \rangle_{\rho} := \operatorname{trace} (\hat{\rho} \hat{A})
  = \sum_k \rho_k \langle \psi_k | \hat{A} | \psi_k \rangle
  = \sum_k \rho_k \langle \hat{A} \rangle_{\psi_k}.
\end{displaymath}

\section{An Example of Dispersion--Free States}

In \cite[p.448]{Bell1966} we find the discussion of a spin-$\frac{1}{2}$ system. In particular we look at the action of the operator
\begin{displaymath}
  \alpha \mathbbm{1} + \bm{\beta} \cdot \bm{\sigma}
\end{displaymath}
on a spin--$\frac{1}{2}$ vector $\psi$. Here $\mathbbm{1}$ is the 2--dimensional unit matrix, and $\bm{\sigma}$ is the vector whose components comprise the individual Pauli spin matrices:
\begin{align*}
  \sigma_1 &:= \begin{pmatrix}
                 0 & 1 \\
                 1 & 0
               \end{pmatrix},
              &
  \sigma_2 &:= \begin{pmatrix}
                 0 & -i \\
                 i & 0
               \end{pmatrix},
              &
  \sigma_3 &:= \begin{pmatrix}
                 1 & 0 \\
                 0 & -1
               \end{pmatrix}.
\end{align*}
$\bm{\beta}$ is some 3--dimensional vector of coefficients.

\begin{prop}[Eigenvalues]
  The operator $\alpha \mathbbm{1} + \bm{\beta} \cdot \bm{\sigma}$ operating on a spin--$\frac{1}{2}$ vector $\psi$ has eigenvalues
  \begin{displaymath}
    \alpha \pm |\bm{\beta}|.
  \end{displaymath}
\end{prop}

\begin{proof}[Calculation of eigenvalues]
  We seek to solve the following equation:
  \begin{displaymath}
    (\alpha \mathbbm{1} + \bm{\beta} \cdot \bm{\sigma}) \psi = \lambda \psi
  \end{displaymath}
  for some complex number $\lambda$. That is, we look for
  \begin{displaymath}
    (\alpha \mathbbm{1} + \bm{\beta} \cdot \bm{\sigma} - \lambda \mathbbm{1}) \psi = 0.
  \end{displaymath}
  Thus we are looking for the nullity of the operator in parentheses. The corresponding eigenvalues are given by the zeroes of the determinant. Thus we look for $\lambda$ satisfying
  \begin{displaymath}
    \det [(\alpha - \lambda) \mathbbm{1} + \bm{\beta} \cdot \bm{\sigma}] \heq 0.
  \end{displaymath}
  In components, the given operator is
  \begin{displaymath}
    \begin{pmatrix}
      \alpha + \beta_3 - \lambda & \beta_1 - i \beta_2 \\
      \beta_1 + i \beta_2 & \alpha - \beta_3 - \lambda
    \end{pmatrix}.
  \end{displaymath}
  The characteristic equation therefore yields
  \begin{displaymath}
    \begin{split}
      0 &\heq (\alpha + \beta_3 - \lambda)(\alpha - \beta_3 - \lambda) - (\beta_1 + i \beta_2) (\beta_1 - i \beta_2) \\
      &= (\alpha + \beta_3)(\alpha - \beta_3) - (\alpha + \beta_3) \lambda - (\alpha - \beta_3) \lambda + \lambda^2 - \left[\beta_1^2 - (i\beta_2)^2 \right] \\
      &= \lambda^2 -2\alpha\lambda + \left(\alpha^2 - |\bm{\beta}|^2\right).
    \end{split}
  \end{displaymath}
  Applying the quadratic formula, this leaves us with
  \begin{displaymath}
    \lambda = \frac{2\alpha \pm \sqrt{4\alpha^2 - 4 \cdot 1 \cdot \left(\alpha^2 - |\bm{\beta}|^2\right)}}{2 \cdot 1}
    = \alpha \pm |\bm{\beta}|,
  \end{displaymath}
  as desired.
\end{proof}

\begin{prop}[Eigenvectors]
  The normalized eigenvectors of the operator $\alpha \mathbbm{1} + \bm{\beta} \cdot
\bm{\sigma}$ are \todo{Double check components... see calculation}
  \begin{align*}
    \phi_{+}
    &:= \frac{1}{\sqrt{2|\bm{\beta}| (|\bm{\beta}|+\beta_z)}}
    \begin{pmatrix}
      |\bm{\beta}|+\beta_z \\
      \beta_x+i\beta_y
    \end{pmatrix},
    &
    \phi_{-}
    &:= \frac{1}{\sqrt{2|\bm{\beta}|(|\bm{\beta}|-\beta_z)}}
    \begin{pmatrix}
      \beta_z - |\bm{\beta}|\\
      \beta_x+i\beta_y
    \end{pmatrix}.
  \end{align*}
\end{prop}

\begin{proof}
  We know the operator has eigenvalues $\lambda_{\pm} := \alpha \pm
|\bm{\beta}|$. We seek to solve the equation
  \begin{displaymath}
    \left( \alpha \mathbbm{1} + \bm{\beta} \cdot
  \bm{\sigma} \right) \phi_{\pm} = \lambda_{\pm} \phi_{\pm},
  \end{displaymath}
  or equivalently
  \begin{displaymath}
    \left[ \left( \alpha - \lambda_{\pm} \right) \mathbbm{1} + \bm{\beta} \cdot
  \bm{\sigma} \right] \phi_{\pm} = 0,
  \end{displaymath}
  for the vector $\phi_{\pm}$ associated with each eigenvalue. In components this gives
  \begin{displaymath}
    \begin{pmatrix}
      \alpha + \beta_z - \lambda_{\pm} & \beta_x - i \beta_y \\
      \beta_x + i \beta_y & \alpha - \beta_z - \lambda_{\pm}
    \end{pmatrix}
    \begin{pmatrix}
      a_{\pm} \\
      b_{\pm}
    \end{pmatrix}
    =
    \begin{pmatrix}
      0 \\
      0
    \end{pmatrix},
  \end{displaymath}
  which leads to the following pair of linear equations:
  \begin{alignat*}{3}
    (\alpha + \beta_z - \lambda_{\pm})&a_{\pm} + (\beta_x - i \beta_y)&b_{\pm} &= 0, \\
    (\beta_x + i \beta_y)&a_{\pm} + (\alpha - \beta_z - \lambda_{\pm})&b_{\pm} &= 0
  \end{alignat*}
  for the components $a_{\pm}, b_{\pm}$ of the spinors $\phi_{\pm}$. The first equation yields the relation
  \begin{displaymath}
    a_{\pm} = - \frac{\beta_x - i \beta_y}{\alpha + \beta_z - \lambda_{\pm}} b_{\pm}.
  \end{displaymath}
  Substituting this relation, the second equation becomes
  \begin{displaymath}
    \begin{split}
      0
      &= \left[ (\alpha - \beta_z - \lambda_{\pm}) - \frac{(\beta_x + i \beta_y)(\beta_x - i \beta_y)}{\alpha + \beta_z - \lambda_{\pm}} \right] b_{\pm} \\
      &= \left[ -(\beta_z \pm |\bm{\beta}|) - \frac{\beta_x^2 + \beta_y^2}{\beta_z \mp |\bm{\beta}|} \right] b_{\pm} \\
      &= - \left[ \frac{(\beta_z \pm |\bm{\beta}|)(\beta_z \mp |\bm{\beta}|) + \beta_x^2 + \beta_y^2}{\beta_z \mp |\bm{\beta}|} \right] b_{\pm} \\
      &= - \left[ \frac{\beta_z^2 - |\bm{\beta}|^2 + \beta_x^2 + \beta_y^2}{\beta_z \mp |\bm{\beta}|} \right] b_{\pm}.
    \end{split}
  \end{displaymath}
  The term in square brackets is identically zero, leaving $b_{\pm}$ as free parameters. Given the relation between $b_{\pm}$ and $a_{\pm}$, we may choose $b_{\pm} := \beta_z \pm |\bm{\beta}|$ for convenience, yielding
  \begin{displaymath}
    a_{\pm} = - \frac{\beta_x - i \beta_y}{\beta_z \mp |\bm{\beta}|} (\beta_z \pm |\bm{\beta}|)
    = \beta_x - i \beta_y.
  \end{displaymath}
  We can therefore say that the vectors
  \begin{displaymath}
    \begin{pmatrix}
      \beta_x - i \beta_y \\
      \beta_z \pm |\bm{\beta}|
    \end{pmatrix}
  \end{displaymath}
  span the eigenspaces of the operator. These vectors have squared norm\footnote{Recall that we are taking the norm of a complex vector, so each component is multiplied by its corresponding complex conjugate.}
  \begin{displaymath}
    \begin{split}
      |\beta_x - i \beta_y|^2 + |\beta_z \pm |\bm{\beta}||^2
      &= \beta_x^2 + \beta_y^2 + \beta_z^2 \pm 2|\bm{\beta}| \beta_z + |\bm{\beta}|^2 \\
      &= 2 |\bm{\beta}|^2 \pm 2 |\bm{\beta}| \beta_z \\
      &= 2 |\bm{\beta}| \left( |\bm{\beta}| \pm \beta_z \right).
    \end{split}
  \end{displaymath}
  Thus the normalized eigenvectors become
  \begin{displaymath}
    \phi_{\pm}
    := \frac{1}{\sqrt{2|\bm{\beta}| (|\bm{\beta}| \pm \beta_z)}}
    \begin{pmatrix}
      \beta_x - i \beta_y \\
      \beta_z \pm |\bm{\beta}|
    \end{pmatrix},
  \end{displaymath}
  as stated.
\end{proof}


\bigskip

The section following the above calculation in \cite{Bell1966} then goes on to introduce a ``hidden variable'' $\lambda$ (unrelated to the $\lambda$ used above). By design, this variable removes any indeterminacy as to which state the spinor $\psi$ is actually in. That is, without the use of the ancillary variable $\lambda$, all we can say is that when we measure the quantity corresponding to the operator $\alpha \mathbbm{1} + \bm{\beta} \cdot \bm{\sigma}$, the measurement must return an eigenvalue of the operator: either $\alpha + |\bm{\beta}|$ or $\alpha - |\bm{\beta}|$, but we don't know which of these will result before making the measurement. However, by incorporating the information of the hidden variable $\lambda$, by virtue of what we mean by the phrase ``hidden variable'', this information should be sufficient to tell us beforehand which of the two possible eigenvalues will result from the measurement. Equation (3) of \cite{Bell1966} is the \emph{specification} of just such a rule: the eigenvalue that results from any measurement will be determined by the value of $\lambda$ according to the rule
\begin{equation}
  \label{eq:hiddenvariablestipulation}
  \alpha + |\bm{\beta}| \operatorname{sgn} \left( \lambda|\bm{\beta}| + \frac{1}{2} |\beta_z| \right) \operatorname{sgn} X,
\end{equation}
where
\begin{displaymath}
  X =
  \begin{cases}
    \beta_z & \text{if $\beta_z \not= 0$},\\
    \beta_x & \text{if $\beta_z = 0$, but $\beta_x \not= 0$},\\
    \beta_y & \text{if $\beta_z = 0$ and $\beta_x =0$}.
  \end{cases}
\end{displaymath}
That is, $X$ is the first non-zero component of $\bm{\beta}$, taken in the order $z, x, y$.

How do we know that $\lambda$ specifies the eigenvalue in the way given by eqn.~(\ref{eq:hiddenvariablestipulation})? We don't. But we're talking completely generally at this point: we're creating a spin--$\frac{1}{2}$ system and asserting that it behaves in the manner given by eqn.~(\ref{eq:hiddenvariablestipulation}). The proper question is in fact, is the stipulation provided by eqn.~(\ref{eq:hiddenvariablestipulation}) an \emph{allowable} stipulation according to the framework of quantum mechanics? To answer that, we must ask a different question: what constraints must this specification satisfy? This seems to be the point behind the calculation that follows eqn.~(3) in \cite{Bell1966}: an average over all variables, including the hidden variables, must give us the proper expectation value.

To calculate the expectation value, we recall that this particular model specifies $-\frac{1}{2} \le \lambda \le \frac{1}{2}$. Then we seek to calculate the average of the quantity in parentheses in eqn.~(\ref{eq:hiddenvariablestipulation}):
\begin{displaymath}
  \int_{-\frac{1}{2}}^{\frac{1}{2}} \operatorname{sgn} \left( \lambda|\bm{\beta}| + \frac{1}{2} |\beta_z| \right) d\lambda.
\end{displaymath}
We note that
\begin{displaymath}
  \lambda|\bm{\beta}| + \frac{1}{2} |\beta_z| \ge 0 \qquad \text{when} \qquad \lambda \ge -\frac{|\beta_z|}{2|\bm{\beta}|}.
\end{displaymath}
Then we have
\begin{displaymath}
  \begin{split}
    \int_{-\frac{1}{2}}^{\frac{1}{2}} \operatorname{sgn} \left( \lambda|\bm{\beta}| + \frac{1}{2} |\beta_z| \right) d\lambda
    &= \int_{-1/2}^{-\frac{|\beta_z|}{2|\bm{\beta}|}} (-1) d\lambda + \int_{-\frac{|\beta_z|}{2|\bm{\beta}|}}^{\frac{1}{2}} (+1) d\lambda \\
    &= \left( \frac{1}{2} + \frac{|\beta_z|}{2|\bm{\beta}|} \right) - \left( -\frac{|\beta_z|}{2|\bm{\beta}|} + \frac{1}{2} \right) \\
    &= \frac{|\beta_z|}{|\bm{\beta}|}.
  \end{split}
\end{displaymath}
Now, given our specification in eqn.~(\ref{eq:hiddenvariablestipulation}), the expectation of the operator $\alpha \mathbbm{1} + \bm{\beta} \cdot \bm{\sigma}$ is given by
\begin{displaymath}
  \begin{split}
    \langle \alpha \mathbbm{1} + \bm{\beta} \cdot \bm{\sigma} \rangle
    &= \int_{-\frac{1}{2}}^{\frac{1}{2}} \left\{ \alpha + |\bm{\beta}| \operatorname{sgn} \left( \lambda|\bm{\beta}| + \frac{1}{2} |\beta_z| \right) \operatorname{sgn} X \right\} d\lambda \\
    &= \alpha + |\bm{\beta}| \operatorname{sgn} X \int_{-\frac{1}{2}}^{\frac{1}{2}} \operatorname{sgn} \left( \lambda|\bm{\beta}| + \frac{1}{2} |\beta_z| \right) d\lambda \\
    &= \alpha + |\bm{\beta}| \operatorname{sgn} X \frac{|\beta_z|}{|\bm{\beta}|} \\
    &= \alpha + \beta_z,
  \end{split}
\end{displaymath}
precisely because \cite{Bell1966} has chosen coordinates where $\psi$ lies along the $z$-axis, so that $X = \beta_z \not= 0$.

It seems that the point here is that, without the hidden variable, the
expectation value of the operator $\alpha \mathbbm{1} + \bm{\beta}
\cdot \bm{\sigma}$ should be $\alpha$, since its eigenvalues are
equally probable, and we have \todo{I think I'm assuming here that the system is in a mixed state of $\psi_{+}$ \& $\psi_{-}$, with $\rho_{\pm} = \frac{1}{2}$. See $\langle \alpha \mathbbm{1} + \bm{\beta} \cdot \bm{\sigma} \rangle_{\psi_{\pm}}$.}
\begin{displaymath}
  \langle \alpha \mathbbm{1} + \bm{\beta} \cdot \bm{\sigma} \rangle
  = \frac{1}{2} \left( \alpha + |\bm{\beta}| \right) + \frac{1}{2} \left( \alpha - |\bm{\beta}| \right)
  = \alpha.
\end{displaymath}
However, by incorporating the hidden variable $\lambda$, not only can we decisively say which eigenvalue will result, but even the expectation value itself is skewed in the direction of the proper state:
\begin{displaymath}
  \langle \alpha \mathbbm{1} + \bm{\beta} \cdot \bm{\sigma} \rangle_{\lambda} = \alpha + \beta_z.
\end{displaymath}
That is, within the standard quantum mechanical formalism (referred to in \cite{Bell1966} as von Neumann's formulation), we can construct a spin--$\frac{1}{2}$ system whereby knowledge of the hidden variable allows the formalism itself to chose unambiguously the \emph{actual} state of the system.

Let us try to unpack a little more the statements made leading up to equation (3) in \cite[p.448]{Bell1966}. In particular Bell states
\begin{quotation}
  \begin{small}
    The dispersion free states are specified by a real number $\lambda$, in the interval $-\frac{1}{2} \le \lambda \le \frac{1}{2}$ as well as the spinor $\psi$. To describe how $\lambda$ determines which eigenvalue the measurement gives, we note that by a rotation of coordinates $\psi$ can be brought to the form
    \begin{displaymath}
      \psi =
      \begin{pmatrix}
        1 \\
        0
      \end{pmatrix}.
    \end{displaymath}
    Let $\beta_x, \beta_y, \beta_z$ be the components of $\bm{\beta}$ in the new coordinate system. Then measurement of $\alpha + \bm{\beta} \cdot \bm{\sigma}$ on the state specified by $\psi$ and $\lambda$ results with certainty in the eigenvalue...
  \end{small}
\end{quotation}
and there follows the expression given above in eqn.~(\ref{eq:hiddenvariablestipulation}). It seems that part of what's being left unsaid is that, from the standpoint of an adherent of a hidden variable theory, the physical system is always \emph{in one particular state} or the other. The notion of being in a superposition of states, e.g.\ $\phi = a_{+} \psi_{+} + a_{-} \psi_{-}$, is a mathematical construct but not supposed to be a reflection of the underlying reality.

Given that perspective on the physical reality, then to decide which
state the physical system is \emph{really} in, we need simply to know
what eigenvalue the measurement will return. If the measurement
returns $\alpha + |\bm{\beta}|$, then the system is in the state
$\psi_{+}$; if $\alpha - |\bm{\beta}|$, then the state $\psi_{-}$. We
can see this directly from the calculation of the expectation value of
the operator $\alpha + \bm{\beta} \cdot \bm{\sigma}$ on the particular
form of the state $\psi$ in Bell's quote above:\todo{Maybe we need to
  see what this calculation gives for all values of $X$.}

\begin{prop}
  The expectation values of the operator $\alpha \mathbbm{1} +
\bm{\beta} \cdot \bm{\sigma}$ in the states $\psi_{+}
:= \begin{pmatrix} 1 \\ 0 \end{pmatrix}$ and $\psi_{-}
:= \begin{pmatrix} 0 \\ 1 \end{pmatrix}$ are $\alpha + \beta_z$ and
  $\alpha - \beta_z$, respectively.
\end{prop}

\begin{proof}
  In the state $\psi_{+}$, the expectation value yields
  \begin{displaymath}
    \begin{split}
      \langle \alpha \mathbbm{1} + \bm{\beta} \cdot \bm{\sigma} \rangle_{\psi_{+}}
      &= \langle \psi_{+} | \alpha \mathbbm{1} + \bm{\beta} \cdot \bm{\sigma} | \psi_{+} \rangle \\
      &=
      \begin{pmatrix}
        1 & 0
      \end{pmatrix}
      \begin{pmatrix}
        \alpha + \beta_z & \beta_x - i \beta_y \\
        \beta_x + i \beta_y & \alpha - \beta_z
      \end{pmatrix}
      \begin{pmatrix}
        1 \\
        0
      \end{pmatrix}
      \\
      &=
      \begin{pmatrix}
        1 & 0
      \end{pmatrix}
      \begin{pmatrix}
        \alpha + \beta_z \\
        \beta_x + i\beta_y
      \end{pmatrix}
      \\
      &= \alpha + \beta_z.
    \end{split}
  \end{displaymath}
  By contrast, for a system in the state $\psi_{-}$ we would have
  \begin{displaymath}
    \begin{split}
      \langle \alpha \mathbbm{1} + \bm{\beta} \cdot \bm{\sigma} \rangle_{\psi_{-}}
      &= \langle \psi_{-} | \alpha \mathbbm{1} + \bm{\beta} \cdot \bm{\sigma} | \psi_{-} \rangle \\
      &=
      \begin{pmatrix}
        0 & 1
      \end{pmatrix}
      \begin{pmatrix}
        \alpha + \beta_z & \beta_x - i \beta_y \\
        \beta_x + i \beta_y & \alpha - \beta_z
      \end{pmatrix}
      \begin{pmatrix}
        0 \\
        1
      \end{pmatrix}
      \\
      &=
      \begin{pmatrix}
        0 & 1
      \end{pmatrix}
      \begin{pmatrix}
        \beta_x - i \beta_y \\
        \alpha - \beta_z
      \end{pmatrix}
      \\
      &= \alpha - \beta_z.
    \end{split}
  \end{displaymath}
\end{proof}

We may view the same calculation from the perspective of the
eigenvectors of the operator $\alpha \mathbbm{1}
+\bm{\beta}\cdot\bm{\sigma}$. In terms of the eigenstates of the operator
$\psi$ may be expanded as follows:
\begin{align*}
  \psi
  :=
  \begin{pmatrix}
    1\\
    0
  \end{pmatrix}
  = \sqrt{\frac{|\bm{\beta}|+\beta_z}{2|\bm{\beta}|}}\phi_{+} - \sqrt{\frac{|\bm{\beta}|-\beta_z}{2|\bm{\beta}|}}\phi_{-}.
\end{align*}
Therefore, the probability that we get $\alpha+|\bm{\beta}|$ is $\dfrac{|\bm{\beta}|+\beta_z}{2|\bm{\beta}|}$, while we get $\alpha-|\bm{\beta}|$ with probability $\dfrac{|\bm{\beta}|-\beta_z}{2|\bm{\beta}|}$.
Thus the expectation value is
\begin{align*}
  \langle \alpha \mathbbm{1} + \bm{\beta} \cdot \bm{\sigma} \rangle_{\psi}
  = \left( \alpha+|\bm{\beta}| \right) \dfrac{|\bm{\beta}|+\beta_z}{2|\bm{\beta}|} + \left( \alpha-|\bm{\beta}| \right) \dfrac{|\bm{\beta}|-\beta_z}{2|\bm{\beta}|}
  = \alpha + \beta_z.
\end{align*}

So in some sense this isn't really anything new, until we add the stipulation that we should be able to say which state the system is in \emph{before} we make a measurement, solely based on our knowledge of the value of the hidden variable $\lambda$.\footnote{Really what we achieve by this construction is that we avoid the necessity of imposing a magical mechanism of ``collapsing the wave function''. Such a collapse is necessitated by the view that the system can be \emph{in} a state which is a superposition of base states. The construction here says that, no, the system was in one of the base states all along, and was never in a superposition.} The preceding calculation assumes we know the state $\psi$ before we take the expectation value, whereas the calculation of $\langle \alpha \mathbbm{1} + \bm{\beta} \cdot \bm{\sigma} \rangle_{\lambda}$ does not.\todo{Is this really right?}

Eqn.~(\ref{eq:hiddenvariablestipulation}) then provides one particular instance for how one could arrange such a hidden variable $\lambda$. Given the value of $\lambda$, then we know from eqn.~(\ref{eq:hiddenvariablestipulation}) which eigenvalue we \emph{will} measure, and so we know what state the system is in \emph{before} we make any measurement. That's how a hidden variable should work. The interesting feature, and the focus of the discussion, is that this particular setup seems permissible within the framework of standard quantum mechanics.\todo[color=green!40]{An example todo note in another color.}

We might also take a moment to note some features of $\lambda$. It seems that Bell falls into the typical practice of physicists which assumes that any random variable is uniformly distributed unless otherwise stated.\footnote{Really there are only two distributions that ever come up in standard presentations of physics: the uniform and the gaussian. Since Bell makes no mention of the gaussian, we fall back to the uniform distribution.} This agrees with the calculation that follows, since the integral expression presented in the calculation for $\langle \alpha \mathbbm{1} + \bm{\beta} \cdot \bm{\sigma} \rangle_{\lambda}$ shows no particular probability density other than the implied constant density.

\section{Refutations of Previous Arguments}

\subsection{von Neumann's Argument}

In \cite[pp.448--449]{Bell1966}, Bell highlights the principal assumption of von Neumann's argument against dispersion--free states:
\begin{quote}
  Any real linear combination of any two Hermitian operators represents an observable, and the same linear combination of expectation values is the expectation value of the combination.
\end{quote}
\emph{Given} this assumption, it is easy to see that dispersion--free states are forbidden.  Since we can always restrict consideration to a two--dimensional subspace, then it suffices to consider a spin--1/2 system.  The construction from the earlier section, then, provides a counterexample.  The expectation value of the operator $\alpha \mathbbm{1} + \bm{\beta} \cdot \bm{\sigma}$ is $\alpha + |\bm{\beta}|$, which is nonlinear in $\bm{\beta}$.  But the eigenvalues of the operator are $\alpha \pm \beta_z$, which are in fact linear in $\bm{\beta}$.  Since dispersion--free states satisfy the requirement that their expectation value always equal one of the eigenvalues, then the above system cannot be allowed within the framework of quantum mechanics according to von Neumann's assumption.

Bell's essential refutation of this argument stems less from the mathematics and rather from the physics under investigation.  The assumption, given two operators corresponding to two measurable quantities, that their linear combination also corresponds to a measurable quantity is a mathematical convenience.  Moreover, it yields the right results in the \emph{linear theory} embodied in the Hilbert space structure.  But it is a transition that is in fact \emph{never realized experimentally}: if we have operators $Q_1$ and $Q_2$, in general the measurement of $Q_1 + Q_2$ requires an experimental setup distinct from that needed to measure either $Q_1$ or $Q_2$ separately.  The canonical example is where $Q_1 = \sigma_x$ and $Q_2 = \sigma_y$, where measurement of $\sigma_x, \sigma_y$, and $\sigma_x + \sigma_y$ require \emph{three}, not two, different orientations of appropriate Stern--Gerlach devices.  Bell states
\begin{quote}
  But this explanation of the nonadditivity of allowed values also establishes the nontriviality of the additivity of expectation values. The latter is a quite peculiar property of quantum mechanical states, not to be expected \emph{a priori}. There is no reason to demand it individually of the hypothetical dispersion free states, whose function it is to reproduce the \emph{measurable} peculiarities of quantum mechanics \emph{when averaged over}.\footnote{Emphasis in the original.}
\end{quote}

It seems that, in some sense, a hidden variable theory is to be viewed as a fully nonlinear theory, akin to that of classical mechanics.  It should not be expected to behave in the same way as the linear quantum theory, except when the hidden variable theory is considered in a statistical manner, and then compared with the fundamentally statistical quantum theory.

\begin{table}
  \label{tab:vonneumann}
  \caption{A comparison of the predictive statements of quantum mechanics and hidden variable theories.}
  \begin{center}
    \begin{tabular}{ccc}
      \toprule
      & \textbf{Expectation} & \textbf{Observation} \\
      & Any number & An eigenvalue of $Q$ \\
      \midrule
      Quantum Mechanics & $\dfrac{\langle \psi | Q | \psi \rangle}{\langle \psi | \psi \rangle}$ & $q_i$ with probability $|a_i|^2$\\
      Hidden Variable & Average over $\lambda$ & $\lambda$ determines which $q_i$ \\
      \bottomrule
    \end{tabular}
  \end{center}
\end{table}

For example, let $|\psi\rangle=\sum_{i}a_i|q_i\rangle$ and $Q|q_i\rangle = q_i |q_i\rangle$.  von Neumann's argument compares the upper left and the lower right of Figure~\ref{tab:vonneumann}. The expectation value and the observed values should not be mixed up. The quantum mechanical {\it expectation value} is linear in $\alpha$ and $\beta_i$, while the {\it observed value}, $\alpha \pm |\bm{\beta}|$ is not linear in them, but there is nothing wrong. What should be compared are the quantum mechanical expectation value and the averaged value over $\lambda$. The hidden variable theory is built up so that it gives the same results as standard quantum mechanics \emph{on average}.  The unique feature of the quantum theory is that \emph{it has no \textbf{non-averaged} underlying theory}.  Therefore it makes no sense to compare quantum mechanics and hidden variable theories until the latter has been averaged over.


\subsection{The Argument of Jauch \& Piron}

\subsubsection{Basics of Projection Operators}

A projection operator $P$ is an operator on a vector space such that
$P^2 = P$. The zero and the unit operator obviously satisfy this
condition.  Any operator of the form
\begin{displaymath}
  A:= \tfrac{1}{2} + \tfrac{1}{2} \hat{\bm{\alpha}} \cdot \bm{\sigma},
\end{displaymath}
where $\hat{\bm{\alpha}}$ is a unit vector, also satisfies that condition, as follows:
\begin{align*}
  A^2
  &:= \left( \tfrac{1}{2} + \tfrac{1}{2} \hat{\bm{\alpha}} \cdot \bm{\sigma} \right)
  \left( \tfrac{1}{2} + \tfrac{1}{2} \hat{\bm{\alpha}} \cdot \bm{\sigma} \right) \\
  &= \tfrac{1}{4} + \tfrac{1}{4} \hat{\alpha}_i \hat{\alpha}_j \sigma_i \sigma_j + \tfrac{1}{2} \hat{\bm{\alpha}} \cdot \bm{\sigma} \\
  &= \tfrac{1}{4} + \tfrac{1}{8} \hat{\alpha}_i \hat{\alpha}_j ( \sigma_i \sigma_j + \sigma_j \sigma_i ) + \tfrac{1}{2} \hat{\bm{\alpha}} \cdot \bm{\sigma} \\
  &= \tfrac{1}{4} + \tfrac{1}{8} \hat{\alpha}_i \hat{\alpha}_j 2 \delta_{ij} + \tfrac{1}{2} \hat{\bm{\alpha}} \cdot \bm{\sigma} \\
  &= \tfrac{1}{4} ( 1 + |\hat{\bm{\alpha}}|^2 ) + \tfrac{1}{2} \hat{\bm{\alpha}} \cdot \bm{\sigma} \\
  &= A.
\end{align*}
%The eigenvalues of $\hat{A}$ are $\tfrac{1}{2}\pm |\bm{\alpha}|$,
%which are 0 and 1, as calculated before.\todo{Does the hat mean
%  $\bm{\alpha}$ is a unit vector? Confusing...}
If we let
\begin{displaymath}
  A_{\pm} := \tfrac{1}{2} \pm \tfrac{1}{2} \hat{\bm{\alpha}} \cdot \bm{\sigma},
\end{displaymath}
then
\begin{displaymath}
  \begin{split}
    A_{+} A_{-}
    &= \left( \tfrac{1}{2} + \tfrac{1}{2} \hat{\bm{\alpha}} \cdot \bm{\sigma} \right)
    \left( \tfrac{1}{2} - \tfrac{1}{2} \hat{\bm{\alpha}} \cdot \bm{\sigma} \right) \\
    &= \tfrac{1}{4} - \tfrac{1}{4} \hat{\bm{\alpha}} \cdot \bm{\sigma} + \tfrac{1}{4} \hat{\bm{\alpha}} \cdot \bm{\sigma} + \tfrac{1}{4} (\hat{\bm{\alpha}} \cdot \bm{\sigma})^2 \\
    &= \tfrac{1}{4} + \tfrac{1}{4} \hat{\bm{\alpha}} \cdot \bm{\sigma} - \tfrac{1}{4} \hat{\bm{\alpha}} \cdot \bm{\sigma} + \tfrac{1}{4} (\hat{\bm{\alpha}} \cdot \bm{\sigma})^2 \\
    &= \left( \tfrac{1}{2} - \tfrac{1}{2} \hat{\bm{\alpha}} \cdot \bm{\sigma} \right)
    \left( \tfrac{1}{2} + \tfrac{1}{2} \hat{\bm{\alpha}} \cdot \bm{\sigma} \right) \\
    &= A_{-} A_{+}.
  \end{split}
\end{displaymath}
Thus the two projection operators commute.

The projection operator $\tfrac{1}{2} \mathbbm{1} + \tfrac{1}{2} \hat{\bm{\alpha}} \cdot \bm{\sigma}$ is a particular case of $\alpha \mathbbm{1} + \bm{\beta} \cdot \bm{\sigma}$. In fact when $\alpha = \tfrac{1}{2}$ and $\bm{\beta} = \tfrac{1}{2}\hat{\bm{\alpha}}$, the eigenvalues are $\tfrac{1}{2} \pm \tfrac{1}{2} |\hat{\bm{\alpha}}| = 0,1$, as required for a projection operator.\footnote{The eigenvalues of a projection operator are 0 and 1.  If $\bm{x}$ is an eigenvector of $P$ and $\lambda$ is its corresponding eigenvalue,
\begin{align*}
  P^2 \bm{x} = P \bm{x} \Leftrightarrow (\lambda^2 - \lambda) \bm{x} = 0.
\end{align*}
Therefore $\lambda = 0,1$.}
%The eigenvector with the eigenvalue 1 is
%\begin{align*}
%    \phi_{+}
%    &:= \frac{1}{\sqrt{2 (1+\alpha_z)}}
%    \begin{pmatrix}
%      1 +\alpha_z \\
%      \alpha_x+i\alpha_y
%    \end{pmatrix}.
%\end{align*}
The eigenvectors corresponding to $\lambda_{\pm} = 0,1$ then become
\begin{displaymath}
  \phi_{\pm}
  := \frac{1}{\sqrt{2 (1 \pm \hat{\alpha}_z)}}
  \begin{pmatrix}
    \hat{\alpha}_x - i \hat{\alpha}_y \\
    \hat{\alpha}_z \pm 1
  \end{pmatrix}.
\end{displaymath}


\subsubsection{Assumptions on Projection Operators}

For the projection operators, Jauch \& Piron make the following assumptions:
\begin{enumerate}[label=(\Alph*)]
  \item\label{list:jpA} Expectation values of commuting projection operators are additive.
  \item\label{list:jpB} If, for some state and two projections $a$ and $b$,
  \begin{align*}
    \langle a \rangle = \langle b \rangle = 1,
  \end{align*}
  then for that state 
  \begin{align*}
    \langle a \cap b \rangle = 1.
  \end{align*}
\end{enumerate}
In a dispersion-free state, the expectation value of an operator must be one of its eigenvalues, 0 or 1 for projections. Since $\tfrac{1}{2}\pm\tfrac{1}{2}\bm{\alpha}\cdot\bm{\sigma}$ commute, by \ref{list:jpA}
\begin{align*}
  \langle \tfrac{1}{2} + \tfrac{1}{2} \hat{\bm{\alpha}} \cdot \bm{\sigma} \rangle +
  \langle \tfrac{1}{2} - \tfrac{1}{2} \hat{\bm{\alpha}} \cdot \bm{\sigma} \rangle
  = \langle \tfrac{1}{2} + \tfrac{1}{2} \hat{\bm{\alpha}} \cdot \bm{\sigma} + \tfrac{1}{2} - \tfrac{1}{2} \hat{\bm{\alpha}} \cdot \bm{\sigma} \rangle
  = 1,
\end{align*}
we have that for the dispersion-free state either
\begin{align*}
  \langle \tfrac{1}{2} + \tfrac{1}{2} \hat{\bm{\alpha}} \cdot \bm{\sigma} \rangle = 1
  \qquad \text{or} \qquad
  \langle \tfrac{1}{2} - \tfrac{1}{2} \hat{\bm{\alpha}} \cdot \bm{\sigma} \rangle = 1.
\end{align*}
Let $\hat{\bm{\alpha}}$ and $\hat{\bm{\beta}}$ be any noncollinear unit vectors and
\begin{displaymath}
  a = \tfrac{1}{2} \pm \tfrac{1}{2} \hat{\bm{\alpha}} \cdot \bm{\sigma}
  \qquad \text{and} \qquad
  b = \tfrac{1}{2} \pm \tfrac{1}{2} \hat{\bm{\beta}} \cdot \bm{\sigma},
\end{displaymath}
with the signs chosen so that $\langle a \rangle = \langle b \rangle = 1$. Then \ref{list:jpB} requires $\langle a \cap b \rangle = 1$.  But with $\hat{\bm{\alpha}}$ and $\hat{\bm{\beta}}$ noncollinear, the subspace spanned by 
\begin{align*}
    \phi_{+,\hat{\bm{\alpha}}}
    &:= \frac{1}{\sqrt{2 (1 + \hat{\alpha}_z)}}
  \begin{pmatrix}
    \hat{\alpha}_x - i \hat{\alpha}_y \\
    \hat{\alpha}_z + 1
  \end{pmatrix},
\end{align*}
and that spanned by 
\begin{align*}
    \phi_{+,\hat{\bm{\beta}}}
    &:= \frac{1}{\sqrt{2 (1 + \hat{\beta}_z)}}
  \begin{pmatrix}
    \hat{\beta}_x - i \hat{\beta}_y \\
    \hat{\beta}_z + 1
  \end{pmatrix},
\end{align*}
have the intersection $0$, which means $a \cap b = 0$, so that $\langle a \cap b \rangle = 0$. This contradiction led Jauch and Piron to conclude that the dispersion-free state is impossible. But again, the contradiction comes from the tacit assumption that \ref{list:jpB} holds for the expectation values of dispersion-free states. \ref{list:jpB} does not have to be satisfied for dispesion-free states. We only require \ref{list:jpB} is reproduced after averaging over hidden variables.

\bibliographystyle{plain}
\bibliography{notes}

\end{document}