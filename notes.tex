\documentclass[12pt]{article}

\usepackage{amsmath,amsthm}
\usepackage{bm} % bold-math, for getting boldface Greek letters
\usepackage{bbm} % to get blackboard-bold fonts, including double-stroke 1

\newtheorem{thm}{Theorem}
\newtheorem{lem}[thm]{Lemma} % including [thm] this way keeps Theorem and Lemma on the same numbering system
\newtheorem{prop}[thm]{Proposition}

\newcommand{\heq}{\stackrel{\heartsuit}{=}}

\title{For Whom the Bell Tunnels}
\author{T `n' T}

\begin{document}
\maketitle

In \cite[p.448]{Bell1966} we find the discussion of a spin-$\frac{1}{2}$ system.  In particular we look at the action of the operator
\begin{displaymath}
  \alpha \mathbbm{1} + \bm{\beta} \cdot \bm{\sigma}
\end{displaymath}
on a spin--$\frac{1}{2}$ vector $\psi$.  Here $\mathbbm{1}$ is the 2--dimensional unit matrix, and $\bm{\sigma}$ is the vector whose components comprise the individual Pauli spin matrices:
\begin{align*}
  \sigma_1 &:= \begin{pmatrix}
                 0 & 1 \\
                 1 & 0
               \end{pmatrix},
              \\
  \sigma_2 &:= \begin{pmatrix}
                 0 & -i \\
                 i & 0
               \end{pmatrix},
              \\
  \sigma_3 &:= \begin{pmatrix}
                 1 & 0 \\
                 0 & -1
               \end{pmatrix}.
\end{align*}
$\bm{\beta}$ is some 3--dimensional vector of coefficients.

\begin{prop}[Eigenvalues]
  The operator $\alpha \mathbbm{1} + \bm{\beta} \cdot \bm{\sigma}$ operating on a spin--$\frac{1}{2}$ vector $\psi$ has eigenvalues
  \begin{displaymath}
    \alpha \pm |\bm{\beta}|.
  \end{displaymath}
\end{prop}

\begin{proof}[Calculation of eigenvalues]
  We seek to solve the following equation:
  \begin{displaymath}
    (\alpha \mathbbm{1} + \bm{\beta} \cdot \bm{\sigma}) \psi = \lambda \psi
  \end{displaymath}
  for some complex number $\lambda$.  That is, we look for
  \begin{displaymath}
    (\alpha \mathbbm{1} + \bm{\beta} \cdot \bm{\sigma} - \lambda \mathbbm{1}) \psi = 0.
  \end{displaymath}
  Thus we are looking for the nullity of the operator in parentheses.  The corresponding eigenvalues are given by the zeroes of the determinant.  Thus we look for $\lambda$ satisfying
  \begin{displaymath}
    \det [(\alpha - \lambda) \mathbbm{1} + \bm{\beta} \cdot \bm{\sigma}] \heq 0.
  \end{displaymath}
  In components, the given operator is
  \begin{displaymath}
    \begin{pmatrix}
      \alpha + \beta_3 - \lambda & \beta_1 - i \beta_2 \\
      \beta_1 + i \beta_2 & \alpha - \beta_3 - \lambda
    \end{pmatrix}.
  \end{displaymath}
  The characteristic equation therefore yields
  \begin{displaymath}
    \begin{split}
      0 &\heq (\alpha + \beta_3 - \lambda)(\alpha - \beta_3 - \lambda) - (\beta_1 + i \beta_2) (\beta_1 - i \beta_2) \\
      &= (\alpha + \beta_3)(\alpha - \beta_3) - (\alpha + \beta_3) \lambda - (\alpha - \beta_3) \lambda + \lambda^2 - \left[\beta_1^2 - (i\beta_2)^2 \right] \\
      &= \lambda^2 -2\alpha\lambda + \left(\alpha^2 - |\bm{\beta}|^2\right).
    \end{split}
  \end{displaymath}
  Applying the quadratic formula, this leaves us with
  \begin{displaymath}
    \lambda = \frac{2\alpha \pm \sqrt{4\alpha^2 - 4 \cdot 1 \cdot \left(\alpha^2 - |\bm{\beta}|^2\right)}}{2 \cdot 1}
    = \alpha \pm |\bm{\beta}|,
  \end{displaymath}
  as desired.
\end{proof}

\bigskip

The section following the above calculation in \cite{Bell1966} then goes on to introduce a ``hidden variable'' $\lambda$ (unrelated to the $\lambda$ used above).  By design, this variable removes any indeterminacy as to which state the spinor $\psi$ is actually in.  That is, without the use of the ancillary variable $\lambda$, all we can say is that when we measure the quantity corresponding to the operator $\alpha \mathbbm{1} + \bm{\beta} \cdot \bm{\sigma}$, the measurement must return an eigenvalue of the operator: either $\alpha + |\bm{\beta}|$ or $\alpha - |\bm{\beta}|$, but we don't know which of these will result before making the measurement.  However, by incorporating the information of the hidden variable $\lambda$, by virtue of what we mean by the phrase ``hidden variable'', this information should be sufficient to tell us beforehand which of the two possible eigenvalues will result from the measurement.  Equation (3) of \cite{Bell1966} is the \emph{specification} of just such a rule: the eigenvalue that results from any measurement will be determined by the value of $\lambda$ according to the rule
\begin{equation}
  \label{eq:hiddenvariablestipulation}
  \alpha + |\bm{\beta}| \operatorname{sgn} \left( \lambda|\bm{\beta}| + \frac{1}{2} |\beta_z| \right) \operatorname{sgn} X,
\end{equation}
where
\begin{displaymath}
  X =
  \begin{cases}
    \beta_z & \text{if $\beta_z \not= 0$},\\
    \beta_x & \text{if $\beta_z = 0$, but $\beta_x \not= 0$},\\
    \beta_y & \text{if $\beta_z = 0$ and $\beta_x =0$}.
  \end{cases}
\end{displaymath}
That is, $X$ is the first non-zero component of $\bm{\beta}$, taken in the order $z, x, y$.

How do we know that $\lambda$ specifies the eigenvalue in the way given by eqn.~(\ref{eq:hiddenvariablestipulation})?  We don't.  But we're talking completely generally at this point: we're creating a spin--$\frac{1}{2}$ system and asserting that it behaves in the manner given by eqn.~(\ref{eq:hiddenvariablestipulation}).  The proper question is in fact, is the stipulation provided by eqn.~(\ref{eq:hiddenvariablestipulation}) an \emph{allowable} stipulation according to the framework of quantum mechanics?  To answer that, we must ask a different question: what constraints must this specification satisfy?  This seems to be the point behind the calculation that follows eqn.~(3) in \cite{Bell1966}: an average over all variables, including the hidden variables, must give us the proper expectation value.

To calculate the expectation value, we recall that this particular model specifies $-\frac{1}{2} \le \lambda \le \frac{1}{2}$.  Then we seek to calculate the average of the quantity in parentheses in eqn.~(\ref{eq:hiddenvariablestipulation}):
\begin{displaymath}
  \int_{-\frac{1}{2}}^{\frac{1}{2}} \operatorname{sgn} \left( \lambda|\bm{\beta}| + \frac{1}{2} |\beta_z| \right) d\lambda.
\end{displaymath}
We note that
\begin{displaymath}
  \lambda|\bm{\beta}| + \frac{1}{2} |\beta_z| \ge 0 \qquad \text{when} \qquad \lambda \ge \frac{|\beta_z|}{2|\bm{\beta}|}.
\end{displaymath}
Then we have
\begin{displaymath}
  \begin{split}
    \int_{-\frac{1}{2}}^{\frac{1}{2}} \operatorname{sgn} \left( \lambda|\bm{\beta}| + \frac{1}{2} |\beta_z| \right) d\lambda
    &= \int_{-1/2}^{\frac{|\beta_z|}{2|\bm{\beta}|}} (-1) d\lambda + \int_{\frac{|\beta_z|}{2|\bm{\beta}|}}^{\frac{1}{2}} (+1) d\lambda \\
    &= \left( \frac{1}{2} - \frac{|\beta_z|}{2|\bm{\beta}|} \right) - \left( \frac{|\beta_z|}{2|\bm{\beta}|} + \frac{1}{2} \right) \\
    &= \frac{|\beta_z|}{|\bm{\beta}|}.
  \end{split}
\end{displaymath}
Now, given our specification in eqn.~(\ref{eq:hiddenvariablestipulation}), the expectation of the operator $\alpha \mathbbm{1} + \bm{\beta} \cdot \bm{\sigma}$ is given by
\begin{displaymath}
  \begin{split}
    \langle \alpha \mathbbm{1} + \bm{\beta} \cdot \bm{\sigma} \rangle
    &= \int_{-\frac{1}{2}}^{\frac{1}{2}} \left\{ \alpha + |\bm{\beta}| \operatorname{sgn} \left( \lambda|\bm{\beta}| + \frac{1}{2} |\beta_z| \right) \operatorname{sgn} X \right\} d\lambda \\
    &= \alpha + |\bm{\beta}| \operatorname{sgn} X \int_{-\frac{1}{2}}^{\frac{1}{2}} \operatorname{sgn} \left( \lambda|\bm{\beta}| + \frac{1}{2} |\beta_z| \right) d\lambda \\
    &= \alpha + |\bm{\beta}| \operatorname{sgn} X \frac{|\beta_z|}{|\bm{\beta}|} \\
    &= \alpha + \beta_z,
  \end{split}
\end{displaymath}
precisely because \cite{Bell1966} has chosen coordinates where $\psi$ lies along the $z$-axis, so that $X = \beta_z \not= 0$.

It seems that the point here is that, without the hidden variable, the expectation value of the operator $\alpha \mathbbm{1} + \bm{\beta} \cdot \bm{\sigma}$ should be $\alpha$, since its eigenvalues are equally probable, and we have
\begin{displaymath}
  \langle \alpha \mathbbm{1} + \bm{\beta} \cdot \bm{\sigma} \rangle
  = \frac{1}{2} \left( \alpha + |\bm{\beta}| \right) + \frac{1}{2} \left( \alpha - |\bm{\beta}| \right)
  = \alpha.
\end{displaymath}
However, by incorporating the hidden variable $\lambda$, not only can we decisively say which eigenvalue will result, but even the expectation value itself is skewed in the direction of the proper state:
\begin{displaymath}
  \langle \alpha \mathbbm{1} + \bm{\beta} \cdot \bm{\sigma} \rangle_{\lambda} = \alpha + \beta_z.
\end{displaymath}
That is, within the standard quantum mechanical formalism (referred to in \cite{Bell1966} as von Neumann's formulation), we can construct a spin--$\frac{1}{2}$ system whereby knowledge of the hidden variable allows the formalism itself to chose unambiguously the \emph{actual} state of the system.

\bibliographystyle{plain}
\bibliography{notes}

\end{document}