\documentclass[12pt]{article}

\usepackage{amsmath,amsthm}
\usepackage{bm} % bold-math, for getting boldface Greek letters
\usepackage{bbm} % to get blackboard-bold fonts, including double-stroke 1

\newtheorem{thm}{Theorem}
\newtheorem{lem}[thm]{Lemma} % including [thm] this way keeps Theorem and Lemma on the same numbering system
\newtheorem{prop}[thm]{Proposition}

\newcommand{\heq}{\stackrel{\heartsuit}{=}}

\title{For Whom the Bell Tunnels}
\author{Just Us}

\begin{document}
\maketitle

In \cite[p.448]{Bell1966} we find the discussion of a spin-$\frac{1}{2}$ system.  In particular we look at the action of the operator
\begin{displaymath}
  \alpha \mathbbm{1} + \bm{\beta} \cdot \bm{\sigma}
\end{displaymath}
on a spin--$\frac{1}{2}$ vector $\psi$.  Here $\mathbbm{1}$ is the 2--dimensional unit matrix, and $\bm{\sigma}$ is the vector whose components comprise the individual Pauli spin matrices:
\begin{align*}
  \sigma_1 &:= \begin{pmatrix}
                 0 & 1 \\
                 1 & 0
               \end{pmatrix},
              \\
  \sigma_2 &:= \begin{pmatrix}
                 0 & -i \\
                 i & 0
               \end{pmatrix},
              \\
  \sigma_3 &:= \begin{pmatrix}
                 1 & 0 \\
                 0 & -1
               \end{pmatrix}.
\end{align*}
$\bm{\beta}$ is some 3--dimensional vector of coefficients.

\begin{prop}[Eigenvalues]
  The operator $\alpha \mathbbm{1} + \bm{\beta} \cdot \bm{\sigma}$ operating on a spin--$\frac{1}{2}$ vector $\psi$ has eigenvalues
  \begin{displaymath}
    \alpha \pm |\bm{\beta}|.
  \end{displaymath}
\end{prop}

\begin{proof}[Calculation of eigenvalues]
  We seek to solve the following equation:
  \begin{displaymath}
    (\alpha \mathbbm{1} + \bm{\beta} \cdot \bm{\sigma}) \psi = \lambda \psi
  \end{displaymath}
  for some complex number $\lambda$.  That is, we look for
  \begin{displaymath}
    (\alpha \mathbbm{1} + \bm{\beta} \cdot \bm{\sigma} - \lambda \mathbbm{1}) \psi = 0.
  \end{displaymath}
  Thus we are looking for the nullity of the operator in parentheses.  The corresponding eigenvalues are given by the zeroes of the determinant.  Thus we look for $\lambda$ satisfying
  \begin{displaymath}
    \det [(\alpha - \lambda) \mathbbm{1} + \bm{\beta} \cdot \bm{\sigma}] \heq 0.
  \end{displaymath}
  In components, the given operator is
  \begin{displaymath}
    \begin{pmatrix}
      \alpha + \beta_3 - \lambda & \beta_1 - i \beta_2 \\
      \beta_1 + i \beta_2 & \alpha - \beta_3 - \lambda
    \end{pmatrix}.
  \end{displaymath}
  The characteristic equation therefore yields
  \begin{displaymath}
    \begin{split}
      0 &\heq (\alpha + \beta_3 - \lambda)(\alpha - \beta_3 - \lambda) - (\beta_1 + i \beta_2) (\beta_1 - i \beta_2) \\
      &= (\alpha + \beta_3)(\alpha - \beta_3) - (\alpha + \beta_3) \lambda - (\alpha - \beta_3) \lambda + \lambda^2 - \left[\beta_1^2 - (i\beta_2)^2 \right] \\
      &= \lambda^2 -2\alpha\lambda + \left(\alpha^2 - |\bm{\beta}|^2\right).
    \end{split}
  \end{displaymath}
  Applying the quadratic formula, this leaves us with
  \begin{displaymath}
    \lambda = \frac{2\alpha \pm \sqrt{4\alpha^2 - 4 \cdot 1 \cdot \left(\alpha^2 - |\bm{\beta}|^2\right)}}{2 \cdot 1}
    = \alpha \pm |\bm{\beta}|,
  \end{displaymath}
  as desired.
\end{proof}

\bibliographystyle{plain}
\bibliography{notes}

\end{document}